% Generated by GrindEQ Word-to-LaTeX 
\documentclass{article} %%% use \documentstyle for old LaTeX compilers

\usepackage[english]{babel} %%% 'french', 'german', 'spanish', 'danish', etc.
\usepackage{amssymb}
\usepackage{amsmath}
\usepackage{txfonts}
\usepackage{mathdots}
\usepackage[classicReIm]{kpfonts}
\usepackage[dvips]{graphicx} %%% use 'pdftex' instead of 'dvips' for PDF output

% You can include more LaTeX packages here 


\begin{document}

%\selectlanguage{english} %%% remove comment delimiter ('%') and select language if required


\noindent 

\noindent 

\noindent 

\noindent 

\noindent 1 

\noindent 4 

\noindent 1 

\noindent 
\section{PRIMATE PERSONALITY TRAIT ASSESSMENT  }

\noindent  

\noindent Primate personality assessments can be made with this questionnaire by assigning a numerical score for all of the personality traits listed on the following pages.  

\noindent Make your judgments on the basis of your own understanding of the trait guided by the short clarifying definition following each trait.  

\noindent The primate's own behaviors and interactions with other primates should be the basis for your numerical ratings.  

\noindent Use your own subjective judgment of typical primate behavior to decide if the primate you are scoring is above, below, or average for a trait. The following seven point scale should be used to make your ratings:\textbf{  }

\noindent \textbf{ }

\begin{enumerate}
\item  \textbf{Displays either total absence or negligible amounts of the trait.  }

\item  \textbf{Displays small amounts of the trait on infrequent occasions.  }

\item  \textbf{Displays somewhat less than average amounts of the trait.  }

\item  \textbf{Displays about average amounts of the trait.  }

\item  \textbf{Displays somewhat greater than average amounts of the trait.  }

\item  \textbf{Displays considerable amounts of the trait on frequent occasions.  }

\item  \textbf{Displays extremely large amounts of the trait.   }
\end{enumerate}

\noindent  

\noindent Please give a rating for each trait even if your judgment seems to be based on a purely subjective impression of the primate and you are somewhat unsure about it. Indicate your rating by placing a cross in the box underneath the chosen number.  

\noindent  

\noindent ? 

\noindent  

\noindent \textbf{Finally, do not discuss your rating of any particular primate with anyone else. As explained in the handout accompanying this questionnaire, this restriction is necessary in order to obtain valid reliability coefficients for the traits. }  

\noindent 
\section{PRIMATE PERSONALITY TRAIT ASSESSMENT }

\noindent Primate's full name: \_\_\_\_\_\_\_\_\_\_\_\_\_\_\_\_  Date (Day/Month/Year): \_\_\_\_\_\_\_\_\_\_ 

\noindent Rater's full name: \_\_\_\_\_\_\_\_\_\_\_\_\_\_\_\_\_\_ 

\noindent \textbf{FEARFUL:} Subject reacts excessively to real or imagined threats by displaying behaviors such as screaming, grimacing, running away or other signs of anxiety or distress. 

\noindent  

\noindent least                            most

  1    2    3    4    5    6    7   

  ?    ?    ?    ?    ?    ?    ?   

\noindent  

\noindent \textbf{DOMINANT:} Subject is able to displace, threaten, or take food from other primates. Or subject may express high status by decisively intervening in social interactions. 

\noindent  

\noindent least                            most

  1    2    3    4    5    6    7   

  ?    ?    ?    ?    ?    ?    ?   

\noindent  

\noindent \textbf{PERSISTENT:} Subject tends to continue in a course of action, task, or strategy for a long time or continues despite opposition from other primates. 

\noindent  

\noindent least                            most

  1    2    3    4    5    6    7   

  ?    ?    ?    ?    ?    ?    ?   

\noindent \textbf{ }

\noindent \textbf{CAUTIOUS:} Subject often seems attentive to possible harm or danger from its actions. Subject avoids risky behaviors. 

\noindent  

\noindent least                            most

  1    2    3    4    5    6    7   

  ?    ?    ?    ?    ?    ?    ?   

\noindent  

\noindent \textbf{STABLE:} Subject reacts to its environment including the behavior of other primates in a calm, equable, way. Subject is not easily upset by the behaviors of other primates. 

\noindent  

\noindent least                            most

  1    2    3    4    5    6    7   

  ?    ?    ?    ?    ?    ?    ?   

\noindent \textbf{ }

\noindent \textbf{AUTISTIC:} Subject often displays repeated, continuous, and stereotyped behaviors such as rocking or self clasping. 

\noindent  

\noindent least                            most

  1    2    3    4    5    6    7   

  ?    ?    ?    ?    ?    ?    ?   

\noindent  

\noindent \textbf{CURIOUS:} Subject has a desire to see or know about objects, devices, or other primates. This includes a desire to know about the affairs of other primates that do not directly concern the subject. 

\noindent  

\noindent least                            most

  1    2    3    4    5    6    7   

  ?    ?    ?    ?    ?    ?    ?   

\noindent \textbf{ }

\noindent \textbf{THOUGHTLESS:} Subject often behaves in a way that seems imprudent or forgetful. 

\noindent  

\noindent least                            most

  1    2    3    4    5    6    7   

  ?    ?    ?    ?    ?    ?    ?   

\noindent  

\noindent \textbf{STINGY/GREEDY:} Subject is excessively desirous or covetous of food, favoured locations, or other resources in the environment. Subject is unwilling to share these resources with others. 

\noindent  

\noindent least                            most

  1    2    3    4    5    6    7   

  ?    ?    ?    ?    ?    ?    ?   

\noindent  

\noindent \textbf{JEALOUS:} Subject is often troubled by others who are in a desirable or advantageous situation such as having food, a choice location, or access to social groups. Subject may attempt to disrupt activities of advantaged primates. 

\noindent  

\noindent least                            most

  1    2    3    4    5    6    7   

  ?    ?    ?    ?    ?    ?    ?   

\noindent  

\noindent \textbf{INDIVIDUALISTIC:} Subject's behavior stands out compared to that of the other individuals in the group. This does not mean that it does not fit or is incompatible with the group. 

\noindent  

\noindent least                            most

  1    2    3    4    5    6    7   

  ?    ?    ?    ?    ?    ?    ?   

\noindent \textbf{ }

\noindent \textbf{RECKLESS:} Subject is rash or unconcerned about the consequences of its behaviors. 

\noindent  

\noindent least                            most

  1    2    3    4    5    6    7   

  ?    ?    ?    ?    ?    ?    ?   

\noindent  

\noindent \textbf{SOCIABLE:} Subject seeks and enjoys the company of other primates and engages in amicable, affable, interactions with them. 

\noindent  

\noindent least                            most

  1    2    3    4    5    6    7   

  ?    ?    ?    ?    ?    ?    ?   

\noindent  

\noindent \textbf{DISTRACTIBLE:} Subject is easily distracted and has a short attention span. 

\noindent  

\noindent least                            most

  1    2    3    4    5    6    7   

  ?    ?    ?    ?    ?    ?    ?   

\noindent  

\noindent \textbf{TIMID:} Subject lacks self confidence, is easily alarmed and is hesitant to venture into new social or non-social situations. 

\noindent  

\noindent least                            most

  1    2    3    4    5    6    7   

  ?    ?    ?    ?    ?    ?    ?   

\noindent \textbf{ }

\noindent \textbf{SYMPATHETIC:} Subject seems to be considerate and kind towards others as if sharing their feelings or trying to provide reassurance. 

\noindent  

\noindent least                            most

  1    2    3    4    5    6    7   

  ?    ?    ?    ?    ?    ?    ?   

\noindent \textbf{ }

\noindent \textbf{PLAYFUL:} Subject is eager to engage in lively, vigorous, sportive, or acrobatic behaviors with or without other primates. 

\noindent  

\noindent least                            most

  1    2    3    4    5    6    7   

  ?    ?    ?    ?    ?    ?    ?   

\noindent  

\textbf{   }

\noindent \textbf{SOLITARY:} Subject prefers to spend considerable time alone not seeking or avoiding contact with other primates. 

\noindent  

\noindent least                            most

  1    2    3    4    5    6    7   

  ?    ?    ?    ?    ?    ?    ?   

\noindent  

\noindent \textbf{VULNERABLE:} Subject is prone to be physically or emotionally hurt as a result of dominance displays, highly assertive behavior, aggression, or attack by another primate. 

\noindent  

\noindent least                            most

  1    2    3    4    5    6    7   

  ?    ?    ?    ?    ?    ?    ?   

\noindent  

\noindent \textbf{INNOVATIVE:} Subject engages in new or different behaviors that may involve the use of objects or materials or ways of interacting with others. 

\noindent  

\noindent least                            most

  1    2    3    4    5    6    7   

  ?    ?    ?    ?    ?    ?    ?   

\noindent \textbf{ }

\noindent \textbf{ACTIVE:} Subject spends little time idle and seems motivated to spend considerable time either moving around or engaging in some overt, energetic behavior. 

\noindent  

\noindent least                            most

  1    2    3    4    5    6    7   

  ?    ?    ?    ?    ?    ?    ?   

\noindent \textbf{HELPFUL:} Subject is willing to assist, accommodate, or cooperate with other primates. 

\noindent  

\noindent least                            most

  1    2    3    4    5    6    7   

  ?    ?    ?    ?    ?    ?    ?   

\noindent \textbf{ }

\noindent \textbf{BULLYING:} Subject is overbearing and intimidating towards younger or lower ranking primates. 

\noindent  

\noindent least                            most

  1    2    3    4    5    6    7   

  ?    ?    ?    ?    ?    ?    ?   

\noindent  

\textbf{   }

\noindent \textbf{AGGRESSIVE:} Subject often initiates fights or other menacing and agonistic encounters with other primates. 

\noindent  

\noindent least                            most

  1    2    3    4    5    6    7   

  ?    ?    ?    ?    ?    ?    ?   

\noindent \textbf{ }

\noindent \textbf{MANIPULATIVE:} Subject is adept at forming social relationships for its own advantage, especially using alliances and friendships to increase its social standing. Primate seems able and willing to use others. 

\noindent  

\noindent least                            most

  1    2    3    4    5    6    7   

  ?    ?    ?    ?    ?    ?    ?   

\noindent  

\noindent \textbf{GENTLE:} Subject responds to others in an easy-going, kind, and considerate manner. Subject is not rough or threatening. 

\noindent  

\noindent least                            most

  1    2    3    4    5    6    7   

  ?    ?    ?    ?    ?    ?    ?   

\noindent  

\noindent \textbf{AFFECTIONATE:} Subject seems to have a warm attachment or closeness with other primates. This may entail frequently grooming, touching, embracing, or lying next to others. 

\noindent  

\noindent least                            most

  1    2    3    4    5    6    7   

  ?    ?    ?    ?    ?    ?    ?   

\noindent  

\noindent \textbf{EXCITABLE:} Subject is easily aroused to an emotional state. Subject becomes highly aroused by situations that would cause less arousal in most primates. 

\noindent  

\noindent least                            most

  1    2    3    4    5    6    7   

  ?    ?    ?    ?    ?    ?    ?   

\noindent \textbf{ }

\noindent \textbf{IMPULSIVE:} Subject often displays some spontaneous or sudden behavior that could not have been anticipated. There often seems to be some emotional reason behind the sudden behavior. 

\noindent  

\noindent least                            most

  1    2    3    4    5    6    7   

  ?    ?    ?    ?    ?    ?    ?   

\noindent  

\noindent \textbf{INQUISITIVE:} Subject seems drawn to new situations, objects, or animals. Subject behaves as if it wishes to learn more about other primates, objects, or persons within its view. 

\noindent  

\noindent least                            most

  1    2    3    4    5    6    7   

  ?    ?    ?    ?    ?    ?    ?   

\noindent  

\noindent \textbf{SUBMISSIVE:} Subject often gives in or yields to another primate. Subject acts as if it is subordinate or of lower rank than other primates. 

\noindent  

\noindent least                            most

  1    2    3    4    5    6    7   

  ?    ?    ?    ?    ?    ?    ?   

\noindent  

\noindent \textbf{COOL:} Subject seems unaffected by emotions and is usually undisturbed, assured, and calm. 

\noindent  

\noindent least                            most

  1    2    3    4    5    6    7   

  ?    ?    ?    ?    ?    ?    ?   

\noindent \textbf{ }

\noindent \textbf{DEPENDENT/FOLLOWER:} Subject often relies on other primates for leadership, reassurance, touching, embracing and other forms of social support. 

\noindent  

\noindent least                            most

  1    2    3    4    5    6    7   

  ?    ?    ?    ?    ?    ?    ?   

\noindent  

\noindent \textbf{IRRITABLE:} Subject often seems in a bad mood or is impatient and easily provoked to anger exasperation and consequent agonistic behavior. 

\noindent  

\noindent least                            most

  1    2    3    4    5    6    7   

  ?    ?    ?    ?    ?    ?    ?   

\noindent \textbf{ }

\noindent \textbf{UNPERCEPTIVE:} Subject is slow to respond or understand moods, dispositions, or behaviors of others. 

\noindent  

\noindent least                            most

  1    2    3    4    5    6    7   

  ?    ?    ?    ?    ?    ?    ?   

\noindent  

\noindent \textbf{PREDICTABLE:} Subject's behavior is consistent and steady over extended periods of time. Subject does little that is unexpected or deviates from its usual behavioral routine. 

\noindent  

\noindent least                            most

  1    2    3    4    5    6    7   

  ?    ?    ?    ?    ?    ?    ?   

\noindent \textbf{ }

\noindent \textbf{DECISIVE:} Subject is deliberate, determined, and purposeful in its activities. 

\noindent  

\noindent least                            most

  1    2    3    4    5    6    7   

  ?    ?    ?    ?    ?    ?    ?   

\noindent  

\noindent \textbf{DEPRESSED:} Subject does not seek out social interactions with others and often fails to respond to social interactions of other primates. Subject often appears isolated, withdrawn, sullen, brooding, and has reduced activity. 

\noindent  

\noindent least                            most

  1    2    3    4    5    6    7   

  ?    ?    ?    ?    ?    ?    ?   

\noindent \textbf{ }

\noindent \textbf{CONVENTIONAL:} Subject seems to lack spontaneity or originality. Subject behaves in a consistent manner from day to day and stays well within the social rules of the group. 

\noindent  

\noindent least                            most

  1    2    3    4    5    6    7   

  ?    ?    ?    ?    ?    ?    ?   

\noindent \textbf{ }

\noindent \textbf{SENSITIVE:} Subject is able to understand or read the mood, disposition, feelings, or intentions of other primates often on the basis of subtle, minimal cues. 

\noindent  

\noindent least                            most

  1    2    3    4    5    6    7   

  ?    ?    ?    ?    ?    ?    ?   

\noindent \textbf{ }

\noindent \textbf{DEFIANT:} Subject is assertive or contentious in a way inconsistent with the usual dominance order. Subject maintains these actions despite unfavourable consequences or threats from others. 

\noindent  

\noindent least                            most

  1    2    3    4    5    6    7   

  ?    ?    ?    ?    ?    ?    ?   

\noindent  

\textbf{   }

\noindent \textbf{INTELLIGENT:} Subject is quick and accurate in judging and comprehending both social and non-social situations. Subject is perceptive and discerning about social relationships. 

\noindent  

\noindent least                            most

  1    2    3    4    5    6    7   

  ?    ?    ?    ?    ?    ?    ?   

\noindent \textbf{ }

\noindent \textbf{PROTECTIVE:} Subject shows concern for other primates and often intervenes to prevent harm or annoyance from coming to them. 

\noindent  

\noindent least                            most

  1    2    3    4    5    6    7   

  ?    ?    ?    ?    ?    ?    ?   

\noindent \textbf{ }

\noindent \textbf{QUITTING:} Subject readily stops or gives up activities that have recently been started. 

\noindent  

\noindent least                            most

  1    2    3    4    5    6    7   

  ?    ?    ?    ?    ?    ?    ?   

\noindent \textbf{ }

\noindent \textbf{INVENTIVE:} Subject is more likely than others to do new things including novel social or non-social behaviors. Novel behavior may also include new ways of using devices or materials. 

\noindent \textbf{ }

\noindent least                            most

  1    2    3    4    5    6    7   

  ?    ?    ?    ?    ?    ?    ?   

\noindent \textbf{ }

\noindent \textbf{CLUMSY:} Subject is relatively awkward or uncoordinated during movements including but not limited to walking, acrobatics, and play. 

\noindent  

\noindent least                            most

  1    2    3    4    5    6    7   

  ?    ?    ?    ?    ?    ?    ?   

\noindent  

\noindent \textbf{ERRATIC:} Subject is inconsistent, indefinite, and widely varying in its behavior and moods. 

\noindent  

\noindent least                            most

  1    2    3    4    5    6    7   

  ?    ?    ?    ?    ?    ?    ?   

\noindent  

\textbf{   }

\noindent \textbf{FRIENDLY:} Subject often seeks out contact with other primates for amiable, genial activities. Subject infrequently initiates hostile behaviors towards other primates. 

\noindent  

\noindent least                            most

  1    2    3    4    5    6    7   

  ?    ?    ?    ?    ?    ?    ?   

\noindent  

\noindent \textbf{ANXIOUS:} Subject often seems distressed, troubled, or is in a state of uncertainty. 

\noindent  

\noindent least                            most

  1    2    3    4    5    6    7   

  ?    ?    ?    ?    ?    ?    ?   

\noindent \textbf{ }

\noindent \textbf{LAZY:} Subject is relatively inactive, indolent, or slow moving and avoids energetic activities. 

\noindent  

\noindent least                            most

  1    2    3    4    5    6    7   

  ?    ?    ?    ?    ?    ?    ?   

\noindent  

\noindent \textbf{DISORGANIZED:} Subject is scatter-brained, sloppy, or haphazard in its behavior as if not following a consistent goal. 

\noindent  

\noindent least                            most

  1    2    3    4    5    6    7   

  ?    ?    ?    ?    ?    ?    ?   

\noindent  

\noindent \textbf{UNEMOTIONAL:} Subject is relatively placid and unlikely to become aroused, upset, happy, or sad. 

\noindent  

\noindent least                            most

  1    2    3    4    5    6    7   

  ?    ?    ?    ?    ?    ?    ?   

\noindent \textbf{IMITATIVE:} Subject often mimics, or copies behaviors that it has observed in other primates. 

\noindent  

\noindent least                            most

  1    2    3    4    5    6    7   

  ?    ?    ?    ?    ?    ?    ?   

\noindent  

\noindent \textbf{INDEPENDENT:} Subject is individualistic and determines its own course of action without control or interference from other primates. 

\noindent  

\noindent least                            most

  1    2    3    4    5    6    7   

  ?    ?    ?    ?    ?    ?    ?   


\end{document}

