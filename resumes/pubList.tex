% publication list - Leverhulme


\documentclass[margin,line]{res}

\oddsidemargin -1.5in
\evensidemargin -1.5in
\textwidth=6.5in
\itemsep=0in
\parsep=0in
\topmargin = -30pt
\textheight = 750pt

\newenvironment{list1}{
  \begin{list}{\ding{113}}{%
      \setlength{\itemsep}{0in}
      \setlength{\parsep}{0in} \setlength{\parskip}{0in}
      \setlength{\topsep}{0in} \setlength{\partopsep}{0in} 
      \setlength{\leftmargin}{0.17in}}}{\end{list}}
\newenvironment{list2}{
  \begin{list}{$\bullet$}{%
      \setlength{\itemsep}{0in}
      \setlength{\parsep}{0in} \setlength{\parskip}{0in}
      \setlength{\topsep}{0in} \setlength{\partopsep}{0in} 
      \setlength{\leftmargin}{0.2in}}}{\end{list}}


\begin{document}

Drew Altschul - Most relevant publications



{\bf In preparation}


{\bf Altschul, D.M.} Assertiveness is reliable, distinct from, and has predictive validity beyond the Big 5 personality traits.

%Farrar, B.G., {\bf Altschul, D.M.}, Clayton, N.S., Fischer, J., Harvey, N., Ostojic, L., Troisi, C., Van Der Mescht, V. Hypotheses, statistical inferences and negative evidence in folk physics studies: a registered report.

Weiss, A., Feldblum, J.T., {\bf Altschul, D.M.}, Foerster, S., Collins, D.A., Gilby, I.C., Kamenya, S., Mjungu, D., Wilson, M.L., Goodall, J., Pusey, A.E. Personality variation in wild chimpanzees is maintained by its changing association with rank: 40 years of the chimpanzees of Gombe, Uganda. 

{\bf Submitted}

The Fragile Families Challenge Team, ... {\bf Altschul, D.M.}, ... Measuring the predictability of life outcomes with a scientific mass collaboration. {\it Proceedings of the National Academy of Sciences, 2nd review.}

Wilson, V.A.D., Freeman, H.D., Parr, L.A., LeFevre, C.E., Ochiai, T., Inoue-Murayama, M., Matsuzawa, T., Weiss, A., {\bf Altschul, D.M.} Assessment of facial metrics in chimpanzees: effects of age, sex, and personality. {\it Evolution \& Human Behaviour, 3rd review.}

%{\bf Altschul, D.M.}, Morton, F.B. Guidelines for using extraction methods in data reduction analyses of social relationship structures.  {\it In revision, American Journal of Primatology.}
%Altschul, D.M., Morton, F.B. (2017). Guidelines for using extraction methods in data reduction analyses of social relationship structure in animals.  {\it In revision.}

%{\bf Altschul, D.M.}, Jensen, G.,  Terrace, H. S. Concept learning of ecological and artificial stimuli in rhesus macaques. {\it In revision, Cognition.} (Preprint available at https://peerj.com/manuscripts/4614/ ).

{\bf Altschul, D.M.}, Hopkins, W.D., Weiss, A. Routine biomarkers, allostatic load, and personality compared between humans and captive chimpanzees. {\it  Royal Society Open Science, In revision.}

{\bf Published}

Many Primates, {\bf Altschul, D.M.}, ... Collaborative open science as a way to reproducibility and new insights in primate cognition research: a systematic review. {\it Japanese Psychological Review, Accepted.}

{\bf Altschul, D.M.}, Iveson, M.I., Deary, I.J. Generational differences in loneliness and its psychological and sociodemographic predictors: an exploratory and confirmatory machine learning study. {\it Psychological Medicine, Accepted.}

{\bf Altschul, D.M.}  Leveraging multiple machine learning techniques to predict major life outcomes from a small set of psychological and socioeconomic variables: a combined bottom-up/top-down approach. {\it Socius, In press.}

Many Primates, {\bf Altschul, D.M.}, ... (2019). Establishing an infrastructure for collaboration in primate cognition research. {\it PLOS One, 14}(10): e0223675.

{\bf Altschul, D.M.},* Robinson, L.M.,* Coleman, K. , Capitanio, J., Wilson, V.A.D. (2019). An exploration of the relationships among facial dimensions, age, sex, dominance
status and personality in rhesus macaques (Macaca mulatta). {\it International Journal of Primatology, 4}(40), 532-552.

Wilson, V.A., Guenther, A., Øverli, Ø., Seltmann, M.W., {\bf Altschul, D.M.} (2019). Future Directions for Personality Research: Contributing New Insights to the Understanding of Animal Behavior. {\it Animals 9}(5), 240.

{\bf Altschul, D.M.}, Hopkins, W.D., Herrelko, E.S., Inoue-Murayama, M., Matsuzawa, T., King, J.E., Ross, S.R., Weiss, A. (2018). Personality links with lifespan in chimpanzees. {\it eLife, 7}, e33781.

{\bf Altschul, D.M.}, Jensen, G.,  Terrace, H. S. (2017). Perceptual category learning of photographic and painterly stimuli in rhesus macaques {\it (Macaca mulatta)} and humans.  {\it PLoS One,} 12(9), e0185576.

{\bf Altschul, D.M.}, Wallace, E.K., Sonnweber, R.S., Tomonaga, M., Weiss, A. (2017). Chimpanzee intellect: personality, performance, and motivation with touchscreen tasks. {\it Royal Society Open Science, 4}(5).

{\bf Altschul, D.M.},* Robinson, L.M.,* Wallace, E.K., \'{U}beda, Y., Llorente, M., Machanda, Z., Slocombe, K.E., Leach, M.C., Waran, N.K., Weiss, A. (2017). Chimpanzees with positive welfare are happier, extraverted, and emotionally stable. {\it Applied Animal Behaviour Science}, 191, 90-97

%Wallace, E.K., {\bf Altschul, D.M.,} Pavonetti, S.P., Benti, B., Koerfer, K., Waller, B., Slocombe, K.S. (2017). Is music enriching for group-housed captive chimpanzees?  {\it PLoS One,} 12 (3), e0172672.

%{\bf Altschul, D.M.}, Terrace, H. S., Weiss, A. (2016). Serial cognition and personality in macaques. {\it Animal Behavior and Cognition,} 3(1), 46.

%Weiss, A., {\bf Altschul, D.} (2016). Methods and applications of animal personality research. In J. Call (Ed.), {\it Handbook of Comparative Psychology}. Washington DC: American Psychological Association.

%Jensen, G., {\bf Altschul, D.} (2015). Two perils of binary categorization: why the study of concepts can't afford true/false testing. {\it Frontiers in psychology,} 6.

%Avdagic, E., Jensen, G., {\bf Altschul, D.}, Terrace, H. (2014). Rapid cognitive flexibility of rhesus macaques performing psychophysical task-switching. {\it Animal Cognition}, 17 (3), 619-631.

%Jensen, G., {\bf Altschul, D.}, Danly, E., Terrace, H.S. (2013). Rhesus macaques use a common representation space to solve distinct problems. {\it PLoS One,} 8 (7), e70285.

Jensen, G., {\bf Altschul, D.}, Terrace, H. (2013). Monkeys would rather see and do: preference for agentic control in rhesus macaques. {\it Experimental Brain Research,} 229 (3), 429-442.


\end{document}