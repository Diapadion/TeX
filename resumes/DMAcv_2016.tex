\documentclass[margin,line]{res}

\oddsidemargin -.5in
\evensidemargin -.5in
\textwidth=6.0in
\itemsep=0in
\parsep=0in
\topmargin = -10pt
\textheight = 700pt

\newenvironment{list1}{
  \begin{list}{\ding{113}}{%
      \setlength{\itemsep}{0in}
      \setlength{\parsep}{0in} \setlength{\parskip}{0in}
      \setlength{\topsep}{0in} \setlength{\partopsep}{0in} 
      \setlength{\leftmargin}{0.17in}}}{\end{list}}
\newenvironment{list2}{
  \begin{list}{$\bullet$}{%
      \setlength{\itemsep}{0in}
      \setlength{\parsep}{0in} \setlength{\parskip}{0in}
      \setlength{\topsep}{0in} \setlength{\partopsep}{0in} 
      \setlength{\leftmargin}{0.2in}}}{\end{list}}


\begin{document}

\name{Drew M Altschul \vspace*{.1in}}

\begin{resume}
\section{\sc Contact Information}
\vspace{.05in}
\begin{tabular}{@{}p{3in}p{4in}}
7 George Square  & {\it E-mail:}  d.m.altschul@sms.ed.ac.uk\\            
Edinburgh, UK & {\it Phone:} +44 7961208343 \\    
EH8 9JZ \\
\end{tabular}

%\section{\sc Research Interests}
%Basis for altrusitic behavior in primates, between group interactions in primates, genetic diversity among primates and genetic impact on behavior, ontogeny of language

% \section{\sc Research Interests}
% Behavioral neuroscience, development of new technologies in neuroscience, neuroethology and relations to sociobiology, executive control and the prefrontal cortex, adult neurogenesis in the central nervous system, acoustic signal processing and analysis

\section{\sc Education}

{\bf The University of Edinburgh}, Edinburgh, Scotland\\
\vspace*{-.1in}
\begin{list1}
\item[] PhD candidate in Psychology
\vspace*{.05in}
\end{list1}

{\bf Brandeis University}, Waltham, Massachusetts\\
\vspace*{-.1in}
\begin{list1}
\item[] MS in Neuroscience, 2010
\vspace*{.05in}
%\item[] GPA: 4.5 (out of 5.0)
\end{list1}

{\bf Massachusetts Institute of Technology}, Cambridge, Massachusetts\\
%{\em Department of Brain and Cognitive Sciences} 
\vspace*{-.1in}
\begin{list1}
\item[] BS in Brain and Cognitive Sciences, 2008
\vspace*{.05in}
%\item[] GPA: 4.5 (out of 5.0)
\end{list1}

%\section{\sc Relevant Coursework}
%\begin{list1}
%\item[] Animal Behavior
%\vspace*{.05in}
%\item[] Brain Structure and its Origins
%\vspace*{.05in}
%\item[] Cellular Neuroscience (Graduate)
%\vspace*{.05in}
%\item[] Computational Neuroscience
%\vspace*{.05in}
%\item[] Developmental Neurobiology (Graduate)
%\vspace*{.05in}
%\item[] Microcomputer Project Laboratory
%\vspace*{.05in}
%\item[] Sensation and Perception
%\vspace*{.05in}
%\item[] Systems Neuroscience (Graduate)

%\end{list1}

\vspace{0.1cm}


\section{\sc Research Experience}

{\bf The University of Edinburgh}, Edinburgh, Scotland \\
Department of Psychology

\vspace{-.2cm}
{\em PCD PhD Scholar}, supervised by Dr Alexander Weiss \hfill {\bf September 2013 - present}\\
\vspace{-.3cm}


{\bf Columbia University}, New York, NY \\
Department of Psychology

\vspace{-.2cm}
{\em Lab Manager}, with Prof Herbert S. Terrace \hfill {\bf June 2010 - June 2013}\\
\vspace{-.3cm}

%Manage all aspects of short and long term lab activities, including the design and implementation of research, writing articles, protocols, and presentations, animal handling and management, data analysis, recruiting students and volunteers, and coordinating their duties and projects in the lab.

\vspace{-.1cm}

{\bf The University of Cape Town}, Cape Town, South Africa \\
Cognitive Ethology Research Group/Baboon Research Unit

\vspace{-.2cm}
{\em Field Assistant}, with Dr Rahel Noser \hfill {\bf September 2009 - April 2010}\\
\vspace{-.3cm}

%Worked with a multinational research group to collect data from free ranging baboons surrounding Cape Town in order construct solutions to problems with baboon-human conflict in the region. Everyday work included design and execution of playback experiments, recording animal vocalizations, GPS tracking of foraging patterns, and behavioral and population scanning of entire troops.

\vspace{-.1cm}

{\bf Brandeis University}, Waltham, MA \\
Department of Neuroscience

\vspace{-.2cm}
{\em Rotation Student} \hfill {\bf August 2008 - May 2009}\\
\vspace{-.3cm}

%Practiced behavioral, systems, and cellular neurobiology techniques, developed novel experiments and methods, analyzed behavioral and physiological data. 

\vspace{-.1cm}

{\bf Massachusetts Institute of Technology}, Cambridge, MA \\
Department of Brain and Cognitive Sciences

\vspace{-.2cm}
{\em Research Assistant}, with Prof Earl Miller \hfill {\bf May 2006 - May 2008}\\
\vspace{-.3cm}

%Work included animal handling, behavioral training, task design and data analysis, MRI analysis, electrode array construction and subsequent recording.

\vspace{-.1cm}

{\bf Willamette University}, Salem, OR \\
Department of Chemistry

\vspace{-.2cm}
{\em Research Assistant}, with Dr Jeff Willemsen \hfill {\bf May 2003 - August 2003}\\

\vspace{-.1cm}

%Interned in an organic chemistry research lab, where I performed chemical syntheses of various compounds in the cyclophane family of organic molecules. 

\section{\sc Teaching Experience}

{\bf  The University of Edinburgh}, Edinburgh, Scotland

\vspace{-.3cm}

\vspace{.1cm}
{\em Tutor}, MSc Univariate and Multivariate Statistics \hfill {\bf September 2013 - May 2016}\\

\vspace{-.4cm}
{\em Tutor}, Year 3 Criticial Analysis \hfill {\bf September 2014 - May 2016}\\

\vspace{-.4cm}
{\em Lab Tutor}, Year 1 Psychology \hfill {\bf February 2015 - April 2015}\\

\vspace{-.4cm}
{\em Tutor}, Year 1 Psychology \hfill {\bf September 2013 - May 2014}\\
\vspace{-.3cm}

\vspace{0.5cm}

{\bf Columbia University}, New York, NY

\vspace{-.2cm}
{\em Course Organizer}, Animal Cognition Seminar \hfill {\bf September 2010 - June 2013}\\
\vspace{-.5cm}


{\bf Massachusetts Institute of Technology}, Cambridge, MA

\vspace{-.2cm}
{\em Teaching Assistant}, Animal Behavior \hfill {\bf September 2007 - January 2008}\\
\vspace{-.5cm}

%Served as the students' individual gateway to understanding course material. Duties included grading assignments, writing assignments and quizzes, advising students on term paper topics and research, and being accessible to students.



\vspace{.3cm}

%\section{\sc Skills} 
%\begin{list2}
%\item Programming/Scripting Languages: proficient with Python, R, C, REALbasic, Scheme, and \LaTeX, familiar with Java, JavaScript, HTML, XML, PHP, Basic, MySQL and Pascal
%\item Computer Applications: MATLAB, LabVIEW, SPSS, Adobe Creative Suite, Microsoft Office or equivalent
%\item Operating Systems: Windows, Linux/Unix, Mac OS X
%\item IT: multipurpose server administration, computer hardware assembly and upkeep
%\item Electrical Engineering: analog, digital, and power electronics circuit design, soldering, and wiring
%\item Audio Engineering: recording, editing, and analysis in the studio and field for use in scientific, musical, and audiobook projects
%\item Over-the-Air Radio Engineering: 3 years experience in live broadcasting
%\item Animal Handling: 2 years handling and behavioral training of rhesus macaques, plus additional exposure to rats, ferrets, and wild primates. Responsibilites have included training the animals in complex visual and motor tasks and transporting them to and from housing on a daily basis
%\item Graphic Design: iPhone app icon and layout design, website design consultation
%
%\vspace{.3cm}
%\end{list2}





\section{\sc Outreach} 


\vspace{-.2cm}
{\em Co-organizer}, Communication Roundtable \hfill International Primatological Society,  {\bf August 2016}\\
\vspace{-.5cm}

\vspace{-.2cm}
{\em Organizer}, Big Brother is Nudging You \hfill  Edinburgh International Science Festival,  {\bf April 2016}\\
\vspace{-.5cm}

\vspace{-.2cm}
{\em Presenter}, Science Night at the Zoo \hfill  Edinburgh International Science Festival,  {\bf April 2016}\\
\vspace{-.5cm}

\vspace{-.2cm}
{\em Organizer}, The Great Ape Debate \hfill  Edinburgh International Science Festival,  {\bf April 2015}\\
\vspace{-.5cm}

\vspace{-.2cm}
{\em Demonstrator}, Museum Late  \hfill  National Museum of Scotland,  {\bf February 2015}\\
\vspace{-.5cm}


\vspace{1cm}

\section{\sc Positions Held} 

\vspace{-.2cm}
{\em Tutor Representative}, Department of Psychology \hfill  {\bf October 2015 - present}\\
\vspace{-.5cm}

\vspace{-.2cm}
{\em Member}, Student Committee, American Society of Primatologists \hfill   {\bf July 2015 - present}\\
\vspace{-.5cm}

\vspace{-.2cm}
{\em Coordinator}, Individual Differences Journal Club \hfill   {\bf September 2014 - present}\\
\vspace{-.5cm}


\vspace{1cm}

\section{\sc Awards} 
\begin{list1}

\item[2015] - Great Britain Sasakawa Foundation, Research Training Travel Grant
\item[2014] - British Society for the Psychology of Individual Differences, Best Poster
\item[2008] - Sigma Xi
\item[2008] - Umaer Basha Undergraduate Research Opportunities Fund
\item[2007] - John Reed Undergraduate Research Opportunities Fund

\end{list1}

\vspace{1cm}

\section{\sc Publications} 

Altschul, D.M., Jensen, G.,  Terrace, H. S. (2016). Perceptual category learning of photographic and painterly stimuli in rhesus macaques {\it (Macaca mulatta)} and humans.  {\it Submitted.} (Preprint available at https://peerj.com/preprints/967/ ).

Altschul, D.M., Jensen, G.,  Terrace, H. S. (2016). Concept learning of ecological and artificial stimuli in rhesus macaques. {\it Submitted.} (Preprint available at https://peerj.com/manuscripts/4614/ ).

Morton, F.B., Altschul, D.M. (2016). A commentary on using extraction methods in data reduction analyses to assess social relationship structure in animals.  {\it Submitted.}

Morgan, V., Altschul, D.M., Terrace H.S. (2016). Strategic and graduated metacognitive judgments by monkeys.  {\it Submitted.}

Altschul, D.M., Terrace, H. S., Weiss, A. (2016). Serial cognition and personality in macaques. {\it Animal behavior and cognition,} 3(1), 46.

Weiss, A., Altschul, D. (2016). Methods and applications of animal personality research. In J. Call (Ed.), {\it Handbook of Comparative Psychology}. Washington DC: American Psychological Association.

Jensen, G., Altschul, D. (2015). Two perils of binary categorization: why the study of concepts can't afford true/false testing. {\it Frontiers in psychology,} 6.

Avdagic, E., Jensen, G., Altschul, D., Terrace, H. (2014). Rapid cognitive flexibility of rhesus macaques performing psychophysical task-switching. {\it Animal Cognition,}, 17 (3), 619-631.

Jensen, G., Altschul, D., Danly, E., Terrace, H.S. (2013). Rhesus macaques use a common representation space to solve distinct problems. {\it PLoS One,} 8 (7), e70285

Jensen, G., Altschul, D., Terrace, H. (2013). Monkeys would rather see and do: preference for agentic control in rhesus macaques. {\it Experimental Brain Research,} 229 (3), 429-442

\vspace{1.5cm}

\section{\sc Conference Presentations} 

Altschul, D.M., Jensen, G.,  Terrace, H. S. (2016). Concept learning of ecological and artifical visual stimuli in humans and captive rhesus macaques. Presenting at International Primatological Society Meeting, Chicago, IL, USA.

Altschul, D. M., King, J. E., Inoue-Murayama, M., Ross, S. R., Weiss, A. (2016). Longevity and personality in captive chimpanzees {\it (Pan troglodytes)}. Presenting at Chimpanzee Symposium, Chicago, IL, USA. (Preprint available at https://peerj.com/manuscripts/10810/ ).

Altschul, D.M., Sonnweber, R.S., Wallace, E.K., Weiss, A. (2016). Chimpanzee motivation, personality, and intellect: evidence from touchscreen tasks. Presented at the Scottish Primate Research Group Meeting, Brechin, UK.

Altschul, D.M., Weiss, A. (2015). Individual differences in chimpanzee {\it (Pan troglodytes)} personality, learning, and engagement in touch screen tasks. Presented at the American Society of Primatologists Meeting, Bend, OR, USA. In {\it American Journal of Primatology,} 77, 93. 111 River St, Hoboken, 07030-5774, NJ, USA: Wiley-Blackwell.

Altschul, D.M., Sinn, D., Weiss, A. (2015). A comparative outlook on chimpanzee metabolic health: a personality and biomarker study. Presented at the Scottish Primate Research Group Meeting, Brechin, UK.

Altschul, D.M., Sinn, D., Weiss, A. (2014).  Personality and blood chemistry associations with cardiovascular health in chimpanzees {\it (Pan troglodytes)}. Presented at the American Society of Primatologists Meeting, Decatur, GA, USA. In {\it American Journal of Primatology,} 76, 61. 111 River St, Hoboken, 07030-5774, NJ, USA: Wiley-Blackwell.

Altschul, D.M., Jensen, G., Terrace, H.S. (2014). Perceptual concept learning using ecological and artificial stimuli in monkeys. Presented at the European Conference on Behavioural Biology, Prague, Czech Republic.

Altschul, D.M., Jensen, G., Terrace, H.S. (2014). Perceptual concept learning using ecological and artificial stimuli in monkeys. Presented at the British Society for the Psychology of Individual Differences, London, UK.


\end{resume}
\end{document}




