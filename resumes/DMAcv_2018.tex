\documentclass[margin,line]{res}

\oddsidemargin -.5in
\evensidemargin -.5in
\textwidth=6.0in
\itemsep=0in
\parsep=0in
\topmargin = -10pt
\textheight = 700pt

\newenvironment{list1}{
  \begin{list}{\ding{113}}{%
      \setlength{\itemsep}{0in}
      \setlength{\parsep}{0in} \setlength{\parskip}{0in}
      \setlength{\topsep}{0in} \setlength{\partopsep}{0in} 
      \setlength{\leftmargin}{0.17in}}}{\end{list}}
\newenvironment{list2}{
  \begin{list}{$\bullet$}{%
      \setlength{\itemsep}{0in}
      \setlength{\parsep}{0in} \setlength{\parskip}{0in}
      \setlength{\topsep}{0in} \setlength{\partopsep}{0in} 
      \setlength{\leftmargin}{0.2in}}}{\end{list}}


\begin{document}

\name{Drew M Altschul \vspace*{.1in}}

\begin{resume}
\section{\sc Contact Information}
\vspace{.05in}
\begin{tabular}{@{}p{3in}p{4in}}
7 George Square  & {\it E-mail:}  drew.altschul@ed.ac.uk\\            
Edinburgh, UK & {\it Phone:} +44 7961208343 \\    
EH8 9JZ \\
\end{tabular}

%\section{\sc Research Interests}
%Basis for altrusitic behavior in primates, between group interactions in primates, genetic diversity among primates and genetic impact on behavior, ontogeny of language

% \section{\sc Research Interests}
% Behavioral neuroscience, development of new technologies in neuroscience, neuroethology and relations to sociobiology, executive control and the prefrontal cortex, adult neurogenesis in the central nervous system, acoustic signal processing and analysis

\section{\sc Education}

{\bf The University of Edinburgh}, Edinburgh, Scotland\\
\vspace*{-.1in}
\begin{list1}
\item[] PhD in Psychology, 2017
\vspace*{.05in}
\end{list1}

{\bf Brandeis University}, Waltham, Massachusetts\\
\vspace*{-.1in}
\begin{list1}
\item[] MS in Neuroscience, 2010
\vspace*{.05in}
%\item[] GPA: 4.5 (out of 5.0)
\end{list1}

{\bf Massachusetts Institute of Technology}, Cambridge, Massachusetts\\
%{\em Department of Brain and Cognitive Sciences} 
\vspace*{-.1in}
\begin{list1}
\item[] BS in Brain and Cognitive Sciences, 2008
\vspace*{.05in}
%\item[] GPA: 4.5 (out of 5.0)
\end{list1}

%\section{\sc Relevant Coursework}
%\begin{list1}
%\item[] Animal Behavior
%\vspace*{.05in}
%\item[] Brain Structure and its Origins
%\vspace*{.05in}
%\item[] Cellular Neuroscience (Graduate)
%\vspace*{.05in}
%\item[] Computational Neuroscience
%\vspace*{.05in}
%\item[] Developmental Neurobiology (Graduate)
%\vspace*{.05in}
%\item[] Microcomputer Project Laboratory
%\vspace*{.05in}
%\item[] Sensation and Perception
%\vspace*{.05in}
%\item[] Systems Neuroscience (Graduate)

%\end{list1}

\vspace{0.1cm}


\section{\sc Research Experience}

{\bf The University of Edinburgh}, Edinburgh, Scotland \\
Department of Psychology

\vspace{-.2cm}
{\em Postdoctoral Research Fellow}, with Prof Ian Deary \hfill {\bf August 2017 - present}\\
\vspace{-.1cm}


\vspace{-.2cm}
{\em PCD PhD Scholar}, supervised by Dr Alexander Weiss \hfill {\bf September 2013 - July 2017}\\
\-\hspace{4.75cm} \& Prof Ian Deary
\vspace{-.1cm}


{\bf Columbia University}, New York, NY \\
Department of Psychology

\vspace{-.2cm}
{\em Lab Manager}, with Prof Herbert  Terrace \hfill {\bf June 2010 - June 2013}\\
\vspace{-.3cm}

%Manage all aspects of short and long term lab activities, including the design and implementation of research, writing articles, protocols, and presentations, animal handling and management, data analysis, recruiting students and volunteers, and coordinating their duties and projects in the lab.

\vspace{-.1cm}

{\bf The University of Cape Town}, Cape Town, South Africa \\
Cognitive Ethology Research Group/Baboon Research Unit

\vspace{-.2cm}
{\em Field Researcher}, with Dr Rahel Noser \hfill {\bf September 2009 - April 2010}\\
\vspace{-.3cm}

%Worked with a multinational research group to collect data from free ranging baboons surrounding Cape Town in order construct solutions to problems with baboon-human conflict in the region. Everyday work included design and execution of playback experiments, recording animal vocalizations, GPS tracking of foraging patterns, and behavioral and population scanning of entire troops.

\vspace{-.1cm}

{\bf Brandeis University}, Waltham, MA \\
Department of Neuroscience

\vspace{-.2cm}
{\em Postgraduate Research Student} \hfill {\bf August 2008 - May 2009}\\
\vspace{-.3cm}

%Practiced behavioral, systems, and cellular neurobiology techniques, developed novel experiments and methods, analyzed behavioral and physiological data. 

\vspace{-.1cm}

{\bf Massachusetts Institute of Technology}, Cambridge, MA \\
Department of Brain and Cognitive Sciences

\vspace{-.2cm}
{\em Undergraduate Researcher}, with Prof Earl Miller \hfill {\bf May 2006 - May 2008}\\
\vspace{-.3cm}

%Work included animal handling, behavioral training, task design and data analysis, MRI analysis, electrode array construction and subsequent recording.



\section{\sc Teaching Experience}

{\bf  The University of Edinburgh}, Edinburgh, Scotland

\vspace{-.3cm}

\vspace{.1cm}
{\em Writing Tutor}, PPLS School Writing Centre \hfill {\bf September 2016 - July 2017}\\

\vspace{-.4cm}
{\em Tutor}, MSc Univariate and Multivariate Statistics \hfill {\bf September 2013 - January 2017}\\

\vspace{-.4cm}
{\em Tutor}, Year 3 Criticial Analysis \hfill {\bf September 2014 - January 2017}\\

\vspace{-.4cm}
{\em Lab Tutor}, Year 1 Psychology \hfill {\bf February 2015 - April 2015}\\

\vspace{-.4cm}
{\em Tutor}, Year 1 Psychology \hfill {\bf September 2013 - May 2014}\\
\vspace{-.3cm}

\vspace{0.5cm}

{\bf Columbia University}, New York, NY

\vspace{-.2cm}
{\em Course Organizer}, Animal Cognition Seminar \hfill {\bf September 2010 - June 2013}\\
\vspace{-.5cm}


{\bf Massachusetts Institute of Technology}, Cambridge, MA

\vspace{-.2cm}
{\em Teaching Assistant}, Animal Behavior \hfill {\bf September 2007 - January 2008}\\
\vspace{-.5cm}

%Served as the students' individual gateway to understanding course material. Duties included grading assignments, writing assignments and quizzes, advising students on term paper topics and research, and being accessible to students.



\vspace{.5cm}

%\section{\sc Skills} 
%\begin{list2}
%\item Programming/Scripting Languages: proficient with Python, R, C, REALbasic, Scheme, and \LaTeX, familiar with Java, JavaScript, HTML, XML, PHP, Basic, MySQL and Pascal
%\item Computer Applications: MATLAB, LabVIEW, SPSS, Adobe Creative Suite, Microsoft Office or equivalent
%\item Operating Systems: Windows, Linux/Unix, Mac OS X
%\item IT: multipurpose server administration, computer hardware assembly and upkeep
%\item Electrical Engineering: analog, digital, and power electronics circuit design, soldering, and wiring
%\item Audio Engineering: recording, editing, and analysis in the studio and field for use in scientific, musical, and audiobook projects
%\item Over-the-Air Radio Engineering: 3 years experience in live broadcasting
%\item Animal Handling: 2 years handling and behavioral training of rhesus macaques, plus additional exposure to rats, ferrets, and wild primates. Responsibilites have included training the animals in complex visual and motor tasks and transporting them to and from housing on a daily basis
%\item Graphic Design: iPhone app icon and layout design, website design consultation
%
%\vspace{.3cm}
%\end{list2}





\section{\sc Outreach} 


\vspace{-.2cm}
{\em Co-organizer}, Communication Roundtable \hfill International Primatological Congress,  {\bf August 2016}\\
\vspace{-.5cm}

\vspace{-.2cm}
{\em Organizer}, Big Brother is Nudging You \hfill  Edinburgh International Science Festival,  {\bf April 2016}\\
\vspace{-.5cm}

\vspace{-.2cm}
{\em Presenter}, Science Night at the Zoo \hfill  Edinburgh International Science Festival,  {\bf April 2016}\\
\vspace{-.5cm}

\vspace{-.2cm}
{\em Organizer}, The Great Ape Debate \hfill  Edinburgh International Science Festival,  {\bf April 2015}\\
\vspace{-.5cm}

\vspace{-.2cm}
{\em Demonstrator}, Museum Late  \hfill  National Museum of Scotland,  {\bf February 2015}\\
\vspace{-.5cm}


\vspace{1cm}

\section{\sc Positions Held} 

\vspace{-.2cm}
{\em Member}, Early Career Committee, American Society of Primatologists \hfill   {\bf July 2015 - present}\\
\vspace{-.5cm}

\vspace{-.2cm}
{\em Tutor Representative}, Department of Psychology \hfill  {\bf October 2015 - July 2017}\\
\vspace{-.5cm}

\vspace{-.2cm}
{\em Coordinator}, Individual Differences Journal Club \hfill   {\bf September 2014 - July 2016}\\
\vspace{-.5cm}


\vspace{0.7cm}

\section{\sc Awards} 
\begin{list1}

\item[2016] - Public Library of Science, Publication Assistance Award
\item[2015] - Great Britain Sasakawa Foundation, Research Training Travel Grant
\item[2014] - British Society for the Psychology of Individual Differences, Best Presentation
\item[2013] - Edinburgh Global Research Scholarship
\item[2013] - Principal's Career Development PhD Scholarship
\item[2008] - Sigma Xi, The Scientific Research Honors Society
\item[2008] - Umaer Basha Undergraduate Research Opportunities Fund
\item[2007] - John Reed Undergraduate Research Opportunities Fund

\end{list1}

\vspace{0.7cm}

\section{\sc Publications} 

{\bf In preparation}


{\bf Altschul, D.M.}, Sinn, D., Hopkins, W.D., Weiss, A. Psychological and physiological stress across cultures and species: personality's impact on allostatic load in American and Japanese humans, and chimpanzees. 	

{\bf Altschul, D.M.}, Wraw, C., Der, G., Gale, C., Deary, I.J. Sex differences in weight gain from adolescence to middle-age: using machine learning to highlight cognitive, behavioral, and socioeconomic factors.

Jensen, G., {\bf Altschul, D.M.}, Terrace, H.S. Comparative transfer of serial representation in humans and rhesus macaques.

{\bf Altschul, D.M.}, Wraw, C., Der, G., Gale, C., Deary, I.J. Changes in depression scores across three decades: the relationships with sex, early-life cognitive function, and sociodemographics.

{\bf Altschul, D.M.}, Gale, C., Deary, I.J. Early-life cognitive function, sociodemographics, and phyiscal and mental health in adults of Cebu, Philippines

Wilson, V.A.D., Freeman, H.D., Parr, L.A., LeFevre, C.E., Ochiai, T., Inoue-Murayama, M., Matsuzawa, T., Weiss, A., {\bf Altschul, D.M.}.  Assessment of facial metrics in chimpanzees: effects of age, sex, and personality.

\vspace{0.3cm}

{\bf Submitted}


{\bf Altschul, D.M.}, Hopkins, W.D., Herrelko, E.S., Inoue-Murayama, M., Matsuzawa, T., King, J.E., Ross, S.R., Weiss, A. Personality links to lifespan in chimpanzees. {\it eLife, 2nd revision.} 

{\bf Altschul, D.M.}, Morton, F.B. Guidelines for using extraction methods in data reduction analyses of social relationship structures.  {\it American Journal of Primatology, 2nd revision.}

{\bf Altschul, D.M.}  Leveraging multiple machine learning techniques to predict major life outcomes from a small set of psychological and socioeconomic variables: a combined bottom-up/top-down approach. {\it Socius, 1st Revision.}

{\bf Altschul, D.M.}, Jensen, G.,  Terrace, H. S. Concept learning of ecological and artificial stimuli in rhesus macaques. {\it Cognition, 1st revision.} (Preprint available at https://peerj.com/manuscripts/4614/ ).

Morgan, V.,{\bf Altschul, D.M.}, Terrace H.S. (2017). Strategic and graduated metacognitive judgments by monkeys.  {\it Journal of Comparative Psychology, 1st revision.}

{\bf Altschul, D.M.}, Hopkins, W.D., Weiss, A. Biomarkers and allostatic load compared between humans and captive chimpanzees. {\it  Royal Society Open Science, In review.}

{\bf Altschul, D.M.}, Robinson, L.M., Coleman, K. , Capitanio, J., Wilson, V.A.D. An exploration of the relationships among facial dimensions, age, sex, dominance
status and personality in rhesus macaques (Macaca mulatta). {\it International Journal of Primatology, In review.}

{\bf Altschul, D.M.}, Wraw, C., Der, G., Gale, C., Deary, I.J. Hypertension and predictive interactions between biological sex and cognitive function. {\it International Journal of Epidemiology, Submitted.}


{\bf Published}

{\bf Altschul, D.M.}, Starr, J.M., Deary, I.J. (2018) Cognitive function is associated with later life glycaemia in the Lothian Birth Cohort of 1936. {\it Diabetologia, Accepted}

{\bf Altschul, D.M.}, Wallace, E.K., Sonnweber, R.S., Tomonaga, M., Weiss, A. (2017). Chimpanzee intellect: personality, performance, and motivation with touchscreen tasks. {\it Royal Society Open Science, 4}(5).

{\bf Altschul, D.M.}, Jensen, G.,  Terrace, H. S. (2017). Perceptual category learning of photographic and painterly stimuli in rhesus macaques {\it (Macaca mulatta)} and humans.  {\it PLoS One, 12}(9), e0185576.

{\bf Altschul, D.M.}, Robinson, L.M., Wallace, E.K., \'{U}beda, Y., Llorente, M., Machanda, Z., Slocombe, K.E., Leach, M.C., Waran, N.K., Weiss, A. (2017). Chimpanzees with positive welfare are happier, extraverted, and emotionally stable. {\it Applied Animal Behaviour Science, 191}, 90-97

Wallace, E.K., {\bf Altschul, D.M.}, Pavonetti, S.P., Benti, B., Koerfer, K., Waller, B., Slocombe, K.S. (2017). Is music enriching for group-housed captive chimpanzees?  {\it PLoS One, 12}(3), e0172672.

{\bf Altschul, D.M.}, Terrace, H. S., Weiss, A. (2016). Serial cognition and personality in macaques. {\it Animal Behavior and Cognition,} 3(1), 46.

Weiss, A., {\bf Altschul, D.} (2016). Methods and applications of animal personality research. In J. Call (Ed.), {\it Handbook of Comparative Psychology}. Washington DC: American Psychological Association.

Jensen, G., {\bf Altschul, D.} (2015). Two perils of binary categorization: why the study of concepts can't afford true/false testing. {\it Frontiers in psychology,} 6.

Avdagic, E., Jensen, G., {\bf Altschul, D.}, Terrace, H. (2014). Rapid cognitive flexibility of rhesus macaques performing psychophysical task-switching. {\it Animal Cognition}, 17 (3), 619-631.

Jensen, G., {\bf Altschul, D.}, Danly, E., Terrace, H.S. (2013). Rhesus macaques use a common representation space to solve distinct problems. {\it PLoS One,} 8 (7), e70285.

Jensen, G., {\bf Altschul, D.}, Terrace, H. (2013). Monkeys would rather see and do: preference for agentic control in rhesus macaques. {\it Experimental Brain Research,} 229 (3), 429-442.

\vspace{0.7cm}

\section{\sc Presentations} 

Altschul, D.M., Brosnan, S., Morton, F. B., Beran, M., Parrish, A., Weiss, A. (2017). Personality and self-control in non-human primates. Presented at the Scottish Primate Research Group Meeting, Brechin, UK.

Altschul, D.M. (2016). How to Lasso: Least absolute shrinkage and selection operator. Presented at EdinbR, Edinburgh, UK.

Altschul, D.M., Jensen, G.,  Terrace, H. S. (2016). Concept learning of ecological and artifical visual stimuli in humans and captive rhesus macaques. Presented at International Primatological Society Meeting, Chicago, IL, USA.

Altschul, D. M., King, J. E., Inoue-Murayama, M., Ross, S. R., Weiss, A. (2016). Longevity and personality in captive chimpanzees {\it (Pan troglodytes)}. Presented at Chimpanzee Symposium, Chicago, IL, USA. (Preprint available at https://peerj.com/manuscripts/10810/ ).

Altschul, D.M., Sonnweber, R.S., Wallace, E.K., Weiss, A. (2016). Chimpanzee motivation, personality, and intellect: evidence from touchscreen tasks. Presented at the Scottish Primate Research Group Meeting, Brechin, UK.

Altschul, D.M., Weiss, A. (2015). Individual differences in chimpanzee {\it (Pan troglodytes)} personality, learning, and engagement in touch screen tasks. Presented at the American Society of Primatologists Meeting, Bend, OR, USA. In {\it American Journal of Primatology,} 77, 93. 111 River St, Hoboken, 07030-5774, NJ, USA: Wiley-Blackwell.

Altschul, D.M., Sinn, D., Weiss, A. (2015). A comparative outlook on chimpanzee metabolic health: a personality and biomarker study. Presented at the Scottish Primate Research Group Meeting, Brechin, UK.

Altschul, D.M., Sinn, D., Weiss, A. (2014).  Personality and blood chemistry associations with cardiovascular health in chimpanzees {\it (Pan troglodytes)}. Presented at the American Society of Primatologists Meeting, Decatur, GA, USA. In {\it American Journal of Primatology,} 76, 61. 111 River St, Hoboken, 07030-5774, NJ, USA: Wiley-Blackwell.

Altschul, D.M., Jensen, G., Terrace, H.S. (2014). Perceptual concept learning using ecological and artificial stimuli in monkeys. Presented at the European Conference on Behavioural Biology, Prague, Czech Republic.

Altschul, D.M., Jensen, G., Terrace, H.S. (2014). Perceptual concept learning using ecological and artificial stimuli in monkeys. Presented at the British Society for the Psychology of Individual Differences, London, UK.


\end{resume}
\end{document}




